\documentclass[12pt,a4paper]{article}
\usepackage[utf8]{inputenc}
\usepackage{amsmath,amsfonts,amssymb}
\usepackage{graphicx}
\usepackage{booktabs}
\usepackage{hyperref}
\usepackage{siunitx}
\title{Second- and Third-Harmonic Generation in Nonlinear Optical Media}
\author{Master Student of Physics}
\date{May 26, 2025}

\begin{document}

\maketitle

\begin{abstract}
This report presents an overview of second- and third-harmonic generation (SHG and THG) as key phenomena in nonlinear optics, with emphasis on theoretical foundations, phase-matching techniques, experimental configurations, and applications in spectroscopy and microscopy.
\end{abstract}

\section{Introduction and Motivation}
Nonlinear optical processes extend linear spectroscopy by enabling frequency conversion through intense electromagnetic fields interacting with matter. In particular, \emph{second-harmonic generation} (SHG) and \emph{third-harmonic generation} (THG) provide access to new spectral regions and high-resolution imaging capabilities. Applications include ultrafast laser pulse characterization, biological microscopy, and material property analysis.

\section{Fundamentals of Nonlinear Polarization}
The response of a dielectric medium to an applied electric field $E(t)$ can be expanded as:
\begin{equation}
P(t) = \varepsilon_0\bigl[\chi^{(1)}E(t) + \chi^{(2)}E^2(t) + \chi^{(3)}E^3(t) + \cdots\bigr],
\end{equation}
where $\chi^{(n)}$ denotes the $n$th-order nonlinear susceptibility tensor. The second-order term gives rise to SHG, generating polarization oscillating at $2\omega$, whereas the third-order term produces THG at $3\omega$ and other phenomena such as four-wave mixing.

\subsection{Energy and Momentum Conservation}
Efficient harmonic generation requires satisfaction of conservation laws:
\begin{align}
\text{Energy:} &\quad \hbar n\omega = \sum_i \hbar \omega_i, \
\text{Momentum (phase matching):} &\quad n k(\omega) = k(n\omega) + \Delta k, \quad \Delta k = 0.
\end{align}
Here, $k(\omega)$ is the wavevector at frequency $\omega$ and $\Delta k$ quantifies the phase mismatch.

\section{Second-Harmonic Generation (SHG)}
SHG arises in noncentrosymmetric crystals through the second-order polarization:
\begin{equation}
P_i(2\omega) = \varepsilon_0 \sum_{jk} \chi^{(2)}{ijk} E_j(\omega)E_k(\omega).
\end{equation}
\subsection{Tensor Properties}
The $\chi^{(2)}{ijk}$ tensor has specific nonzero components determined by crystal symmetry. For example, in a LiNbO$3$ crystal:
\begin{table}[h!]
\centering
\begin{tabular}{lc}
\toprule
Component & Value (pm/V) \\
\midrule
$d_{31} = \chi^{(2)}_{311}$ & 4.5 \\
$d_{33} = \chi^{(2)}_{333}$ & 27.0 \\
\bottomrule
\end{tabular}
\caption{Selected nonlinear coefficients for LiNbO$_3$.}
\end{table}

\subsection{Phase-Matching Techniques}
\textbf{Birefringent Phase Matching.} By exploiting crystal birefringence, one can choose ordinary and extraordinary polarizations to satisfy:
\begin{equation}
n k_o(\omega) = k_e(2\omega).
\end{equation}
\textbf{Quasi-Phase Matching.} Periodic poling in materials such as PPLN introduces a modulation of $\chi^{(2)}$ with period $\Lambda$, compensating phase mismatch via reciprocal vectors $G = 2\pi/\Lambda$.

\begin{figure}[h!]
\centering
\includegraphics[width=0.6\textwidth]{shg_phase_matching.png}
\caption{Schematic of quasi-phase-matched SHG in a periodically poled crystal.}
\end{figure}

\section{Third-Harmonic Generation (THG)}
THG is a third-order process present in both centrosymmetric and noncentrosymmetric media, described by:
\begin{equation}
P_i(3\omega) = \varepsilon_0 \sum_{jkl} \chi^{(3)}_{ijkl} E_j(\omega)E_k(\omega)E_l(\omega).
\end{equation}

\subsection{Phase Matching and Cascading}
Perfect phase matching for THG is challenging; one may use
\begin{itemize}
\item \emph{Bulk phase matching}, adjusting dispersion via angle or temperature tuning.
\item \emph{Cascading SHG processes}, where sequential $\chi^{(2)}$ interactions ($\omega \to 2\omega$, then $\omega+2\omega \to 3\omega$) effectively generate THG with enhanced efficiency.
\end{itemize}

\begin{figure}[h!]
\centering
\includegraphics[width=0.6\textwidth]{thg_cascading.png}
\caption{Energy-level diagram illustrating cascaded SHG and sum-frequency mixing to produce THG.}
\end{figure}

\section{Experimental Setup and Considerations}
A typical setup for SHG/THG experiments includes:
\begin{itemize}
\item A femtosecond pulsed laser (e.g., Ti:sapphire, 800nm, 100fs)
\item Beam-shaping and focusing optics (lenses or microscope objectives)
\item Nonlinear crystal mounted on a rotation/temperature-controlled stage
\item Filters or dichroic mirrors to separate fundamental and harmonic beams
\end{itemize}
Key challenges include crystal damage thresholds, beam walk-off, and maintaining spatial overlap. Temperature stabilization is often required for fine-tuning phase matching.

\section{Applications}
\subsection{SHG Microscopy}
SHG provides intrinsic contrast in noncentrosymmetric biological structures (e.g., collagen), enabling label-free imaging.[Placeholder for experimental image]

\subsection{Ultraviolet THG Spectroscopy}
Tripling near-infrared lasers yields UV radiation for high-resolution absorption studies in materials science.

\section{Summary and Outlook}
We have reviewed the physical principles of SHG and THG, emphasizing the roles of $\chi^{(2)}$ and $\chi^{(3)}$ susceptibilities and phase-matching strategies. Future directions include integrated photonic waveguides with engineered dispersion and metasurfaces for enhanced harmonic conversion efficiencies.

\end{document}