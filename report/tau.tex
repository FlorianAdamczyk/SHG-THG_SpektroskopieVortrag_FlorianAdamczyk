%%%%%%%%%%%%%%%%%%%%%%%%%%%%%%%%%%%%%%%%%%%%%%%%%%%%%%%%%%%
% --------------------------------------------------------
% Tau
% LaTeX Template
% Version 2.4.4 (28/02/2025)
%
% Author: 
% Guillermo Jimenez (memo.notess1@gmail.com)
% 
% License:
% Creative Commons CC BY 4.0
% --------------------------------------------------------
%%%%%%%%%%%%%%%%%%%%%%%%%%%%%%%%%%%%%%%%%%%%%%%%%%%%%%%%%%%

\documentclass[9pt,a4paper,twocolumn,twoside]{tau-class/tau}
\usepackage[ngerman]{babel}

%% Draft watermark
% \usepackage{draftwatermark}

% % custom Citation commands
% \DeclareCiteCommand{\citeauthortitle}
%   {\usebibmacro{prenote}}
%   {\usebibmacro{citeindex}%
%    \printnames{labelname}%
%    \setunit{\space\textendash\space}
%    \printfield{title}}
%   {\multicitedelim}
%   {\usebibmacro{postnote}}

%   \DeclareCiteCommand{\citeauthortitleurl}
%   {\usebibmacro{prenote}}
%   {\usebibmacro{citeindex}%
%    \printnames{labelname}%
%    \setunit{\space\textendash\space}
%    \printfield{title}%
%    \setunit{\addsemicolon\space}
%    \printfield{url}}
%   {\multicitedelim}
%   {\usebibmacro{postnote}}

% \DeclareCiteCommand{\parenciteauthortitle}
%   {\usebibmacro{prenote}}
%   {\bibopenparen\usebibmacro{citeindex}%
%    \printnames{labelname}%
%    \setunit{\space\textendash\space}% <- Hier wird das Trennzeichen ":" hinzugefügt
%    \printfield{title}\bibcloseparen}
%   {\multicitedelim}
%   {\usebibmacro{postnote}}

% \makeatletter
% \renewcommand\footnotesize{\tiny}
% \makeatother

% \newcommand{\figcite}[1]{\\[0mm]{\tiny Quelle: \cite{#1}}}
% \newcommand{\figciteweb}[1]{\\[0mm]{\tiny aus: \citeauthortitle{#1}}}
% \newcommand{\figciteweburl}[1]{\\[0mm]{\tiny aus: \citeauthortitleurl{#1}}}

\usepackage{pdfpages}
\usepackage{animate}

%----------------------------------------------------------
% TITEL
%----------------------------------------------------------

\journalname{M.Sc. Physik: Spektroskopie}
\title{Second- und Third-Harmonic Generation (SHG/THG) – Grundlagen, Phasenanpassung und Anwendungen}

%----------------------------------------------------------
% AUTOREN, AFFILIATION UND PROFESSOR
%----------------------------------------------------------

\author[a,1]{Florian Marius Adamczyk}

%----------------------------------------------------------

\affil[a]{Justus-Liebig-Universität Gießen, Institut für Physik, Deutschland}

\professor{PD Dr. Arash Rahimi-Iman, Dipl.-Ing.}

%----------------------------------------------------------
% FUSSZEILE
%----------------------------------------------------------

\institution{Justus-Liebig-Universität Gießen}
\footinfo{SHG \& THG Bericht}
\theday{\today}
\leadauthor{Adamczyk, F.}
\course{M.Sc. Physik: Spektroskopie}

%----------------------------------------------------------
% ABSTRACT UND SCHLAGWÖRTER
%----------------------------------------------------------

\begin{abstract}    
In diesem Bericht werden die Grundlagen der Second- und Third-Harmonic Generation (SHG/THG) in der nichtlinearen Optik vorgestellt. Es wird erklärt, wie nichtlineare Polarisation zur Frequenzverdopplung und -verdreifachung führt, welche Rolle die Materialsymmetrie spielt und wie Phasenanpassung (Phase Matching) die Effizienz beeinflusst. Typische experimentelle Aufbauten, Anwendungen (z.B. Mikroskopie) und zentrale Beispiele werden anhand von Abbildungen erläutert.
\end{abstract}

%----------------------------------------------------------

\keywords{SHG, THG, Nichtlineare Optik, Phasenanpassung, Frequenzverdopplung}

%----------------------------------------------------------

\begin{document}
		
    \maketitle 
    \thispagestyle{firststyle} 
    \tauabstract 
    % \tableofcontents
    % \linenumbers 
    
%----------------------------------------------------------

\section{Einleitung und Motivation}
Nichtlineare optische Prozesse ermöglichen es, neue Frequenzen aus intensiven Laserfeldern zu erzeugen. Besonders die Second-Harmonic Generation (SHG) und Third-Harmonic Generation (THG) sind zentrale Effekte, die in der modernen Lasertechnologie, Mikroskopie und Materialanalyse Anwendung finden.

% \begin{figure}[!ht]
% \centering
% \includegraphics[width=0.6\columnwidth]{figures/experimental.png}
% \caption{Schematischer Aufbau für SHG/THG-Experimente.\figcite{JkwchuiImage}}
% \end{figure}

%----------------------------------------------------------

\section{Grundlagen der nichtlinearen Optik}
Die Polarisation $P$ eines Mediums als Antwort auf ein elektrisches Feld $E$ lässt sich als Potenzreihe schreiben:
\begin{equation}
P = \varepsilon_0\left[\chi^{(1)}E + \chi^{(2)}E^2 + \chi^{(3)}E^3 + \dots\right]
\end{equation}
Die Terme $\chi^{(2)}$ und $\chi^{(3)}$ beschreiben die nichtlinearen Effekte. $\chi^{(2)}$ ist nur in nicht-zentrosymmetrischen Materialien ungleich null (z.B. viele Kristalle), $\chi^{(3)}$ existiert in allen Medien.

\begin{figure}[!ht]
\centering
\includegraphics[width=0.6\columnwidth]{../praes/Images/Fig.1 optics.jpeg}
\caption{Links: Lineare, rechts: nichtlineare Antwort eines Mediums auf Licht.\cite{Science20NonlinearOptics2014}}
\end{figure}

%----------------------------------------------------------

\section{Zweitharmonische Generation (SHG)}
Bei der SHG verschmelzen zwei Photonen der Frequenz $\omega$ zu einem Photon mit $2\omega$. Voraussetzung: $\chi^{(2)} \neq 0$ (keine Inversionssymmetrie).
\begin{equation}
P^{(2)}(2\omega) = \varepsilon_0 \chi^{(2)} E(\omega)^2
\end{equation}

Typische Kristalle: BBO, KTP, KDP, LBO. Effiziente SHG erfordert starke, gepulste Laser und exakte Ausrichtung.

\begin{figure}[!ht]
\centering
\includegraphics[width=0.6\columnwidth]{../praes/Images/Schematic_of_the_SHG_conversion_of_an_excited_wave_in_a_non-linear_medium.png}
\caption{Schematische Darstellung: Einfallende Welle erzeugt zweite Harmonische im nichtlinearen Medium.\cite{BPAegirsson2017}}
\end{figure}

Historisch wurde SHG erstmals 1961 mit einem Rubinlaser beobachtet (Franken et al.).

%----------------------------------------------------------

\section{Drittharmonische Generation (THG)}
Bei der THG verschmelzen drei Photonen $\omega$ zu einem Photon $3\omega$. $\chi^{(3)}$ ist in allen Materialien vorhanden, daher ist THG auch in Flüssigkeiten und Gasen möglich.
\begin{equation}
P^{(3)}(3\omega) = \varepsilon_0 \chi^{(3)} E(\omega)^3
\end{equation}

\begin{figure}[!ht]
\centering
\includegraphics[width=0.95\columnwidth]{../praes/Images/thg.png}
\caption{Drei Photonen $\omega$ erzeugen ein Photon $3\omega$.\cite{Boyd2020}}
\end{figure}

THG ist meist schwächer als SHG, aber besonders effektiv an Grenzflächen oder bei Brechungsindexgradienten (z.B. THG-Mikroskopie).

%----------------------------------------------------------

\section{Nichtlineare Suszeptibilität und Symmetrie}
Ob ein Material SHG zeigt, hängt von der Symmetrie ab. In zentrosymmetrischen Medien verschwindet $\chi^{(2)}$, nur $\chi^{(3)}$ bleibt.

\begin{figure}[!ht]
\centering
\includegraphics[width=0.45\columnwidth]{../praes/Images/pot_noncentrosym.png}
\includegraphics[width=0.45\columnwidth]{../praes/Images/pot_centrosym.png}
\caption{Links: Potential ohne Inversionssymmetrie ($\chi^{(2)} \neq 0$), rechts: mit Inversionssymmetrie ($\chi^{(2)} = 0$).\cite{Boyd2020}}
\end{figure}

%----------------------------------------------------------

\section{Phasenanpassung (Phase Matching)}
Für effiziente Frequenzkonversion müssen Grund- und Harmonische phasenkohärent bleiben. Dies wird durch Doppelbrechung (Birefringenz) oder periodische Polung erreicht.

\begin{figure}[!ht]
\centering
      \animategraphics[autoplay,loop,width=0.6\columnwidth]{10}{../praes/Images/Phase Matching/imperfect}{001}{021}\\[-1mm]{Animation: Imperfektes Phasenmatching – die erzeugte Welle läuft außer Phase.}
    \cite{Bertolotti2019b}
\end{figure}



%----------------------------------------------------------

\section{Experimenteller Aufbau und Anwendungen}
Ein typischer Aufbau besteht aus Laser, nichtlinearem Kristall, Strahlführung, Filtern und Detektor.

\begin{figure}[!ht]
\centering
\includegraphics[width=0.7\columnwidth]{../praes/Images/Experimental-setup-measuring-SHG-and-THG-intensity-relative-to-laser-energy.png}
\caption{Experimenteller Aufbau zur Messung der SHG- und THG-Intensität in Abhängigkeit von der Laserenergie.\cite{ResearchGate2025}}
\end{figure}

\subsection{SHG-Mikroskopie}
SHG-Mikroskopie nutzt die nichtlineare Antwort von Strukturen wie Kollagen für kontrastreiche, markerfreie Bildgebung in der Biologie.

\subsection{Weitere Anwendungen}
\begin{itemize}
    \item Frequenzverdopplung in Lasern (z.B. grüne Laserpointer)
    \item Materialcharakterisierung
    \item Oberflächenanalytik (z.B. SHG an Grenzflächen)
\end{itemize}

%----------------------------------------------------------

\section{Zusammenfassung und Ausblick}
SHG und THG sind fundamentale Prozesse der nichtlinearen Optik. Die Effizienz hängt von Material, Symmetrie und Phasenanpassung ab. Anwendungen reichen von der Lasertechnik bis zur Biophysik.

%----------------------------------------------------------

\section{Komponenten und Messung}
Die wichtigsten Komponenten eines SHG/THG-Experiments sind:
\begin{itemize}
    \item \textbf{Laser:} Pulsdauer, Wellenlänge, Leistung (z.B. Nd:YAG 1064 nm, Ti:Sa 800 nm)
    \item \textbf{Kristall:} Material (BBO, KTP, LBO), Dicke, Temperatur, Ausrichtung
    \item \textbf{Strahlführung:} Polarisation, Fokussierung auf den Kristall
    \item \textbf{Filter:} Blockieren das Grundsignal, lassen nur die Harmonische durch
    \item \textbf{Detektor:} Photodiode oder Spektrometer
\end{itemize}
Die Intensität der erzeugten Harmonischen wird typischerweise gegen die Eingangsleistung oder den Kristallwinkel gemessen. Für SHG ist die Abhängigkeit quadratisch ($I_{SHG} \propto P^2$), für THG kubisch.


%----------------------------------------------------------

\section{Weitere Spezialfälle und Prozesse}
\subsection{Sequential THG und HHG}
Bei Sequential THG erfolgt die Frequenzverdreifachung in mehreren Stufen ($\omega \rightarrow 2\omega \rightarrow 3\omega$). High Harmonic Generation (HHG) ist ein nichtlinearer, nichtperturbativer Prozess, der sehr hohe Ordnungen ermöglicht.

\subsection{Summenfrequenz- und Differenz-Frequenz-Generation (SFG/DFG)}
SFG: Zwei Photonen unterschiedlicher Frequenz ($\omega_1$, $\omega_2$) erzeugen ein Photon mit $\omega_3 = \omega_1 + \omega_2$. Beide Prozesse benötigen starke Felder und Phasenanpassung.

\subsection{SHG vs. Zweiphotonenabsorption (TPA)}
SHG ist ein kohärenter Prozess, bei dem zwei Photonen zu einem neuen Photon verschmelzen. TPA ist ein inkohärenter Prozess, bei dem zwei Photonen gleichzeitig absorbiert werden, um ein Elektron anzuregen.
\begin{figure}[!ht]
\centering
\includegraphics[width=0.6\columnwidth]{../praes/Images/shg vs tpa.png}\\[-1mm]
\caption{{SHG vs. TPA: Prinzipdarstellung, \cite{Zhang2020}}}
\end{figure}

%----------------------------------------------------------

\section{Beispiele und Anwendungen}
\begin{itemize}
    \item \textbf{Grüne Laserpointer:} SHG aus IR-Laser (1064 nm) erzeugt sichtbares Licht (532 nm)
    \item \textbf{Materialcharakterisierung:} Bestimmung von $\chi^{(2)}$ und $\chi^{(3)}$
    \item \textbf{Oberflächenanalytik:} SHG ist sensitiv für nicht-zentrosymmetrische Bereiche (z.B. Grenzflächen, Adsorbate)
    \item \textbf{Biologische Bildgebung:} SHG- und THG-Mikroskopie für label-freie, kontrastreiche Aufnahmen (z.B. Kollagen, Zellmembranen)
\end{itemize}

\begin{figure}[!ht]
\centering
\includegraphics[width=0.6\columnwidth]{../praes/Images/THG-and-SHG-signals-in-the-zebrafish-embryo-during-cleavage-stages-A-Sagittal-THG.png}
\caption{SHG/THG-Mikroskopie am Zebrafisch-Embryo.\cite{ResearchGateZebrafish2009}}
\end{figure}

%----------------------------------------------------------

\printbibliography % Literaturverzeichnis

%----------------------------------------------------------

% \section*{Anhang}
% Im Anhang sind beispielhafte LLM-Konversationen dokumentiert, die zur Erstellung dieses Berichts genutzt wurden.

% \includepdf[pages=-]{LLM_documentation/001.pdf}
% \includepdf[pages=-]{LLM_documentation/+01.pdf}
% \includepdf[pages=-]{LLM_documentation/FirstTest/firstTest.pdf}

\end{document}